\section{Outcomes}

We have seen a variety of positive outcomes from the redesign.  As
already mentioned, the design provides clearer connections to their
post-college careers and work experiences and students feel more
committed to these projects because they understand the potential
impact.  More broadly, they understand that the work they can do
as computer scientists can have real and meaningful value.

We see significant growth in student's soft skills, a particularly
important learning outcome.
Because students must regularly meet with a
non-technical partner, understand the partner's needs, and
describe their technical work to the partner,
they greatly expand their communication skills.
As is often the case with small teams, they develop better interpersonal
skills.  Because we ask them to reflect on and compare their individual
strengths at the start of the semester,
they think more broadly
about those skills and the skills a successful
project requires.  
Because we do
not require that projects be finished in the semester and because
we tell the students to time-box their work, they not only feel
less stress, but are more willing to agree upon appropriate levels
of ambition. 

Ruby on Rails has proven both a strength and a weakness
of the course.  By the end of the semester, when they've started to develop
a better understanding of the Rails model, many students now realize that they 
are able to put together a small, database-backed, Web application
in about a week of full-time work, at least if everything goes as
expected.  
That realization gives them a sense of self-efficacy
and accomplishment.  At the same time, most also understand that if things
don't go right, and they often don't, there are additional challenges
to face.

We hope that these kinds of challenges help them develop broader
skills in learning how to learn.  Most are used to situations in
which the answers can be found in a textbook, from the instructor,
or, too frequently, on Stack Overflow.  Doing design and facing 
unexpected problems leads students to situations in
which they can't immediately find an answer.  What do you do when
Stack Overflow doesn't answer your question?  They must develop skills
at debugging (and in test-driven design that helps them
debug), in searching broadly and on sites like Stack Overflow
to find relevant answers, in analyzing the applicability of advice
that doesn't quite match their problem, and in adapting other
solutions to their situation.  Some clearly develop these kinds
of skills.  

We also hope that these projects help students develop more of a
tolerance of ambiguity.  Most are accustomed to assignments with
clear requirements; some find themselves frustrated that these 
project are open ended and, in many cases, they must develop
their own requirements or, more precisely, translate their clients'
requirements.

Unfortunately, we have not seen as many positive effects of
end-of-course evaluations as we had hoped.  While some students
describe the course as the center of their CS education, others
seem to find it almost pointless.  As one student puts it, ``In this course,
more than most, what you get out of this course depends directly
on what you put into the course.''  We continue to explore ways to
help students better understand and embrace these benefits.

% Add a final paragraph?  Probably not, the paper is still long.
