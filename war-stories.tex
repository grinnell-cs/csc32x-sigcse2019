\section{War stories}

Of course, not everything in the course went smoothly.  In this section,
we describe some of the more significant problems that arose throughout
the years and suggest some lessons that each reveals.

\subsection{Revealing an AWS key}

Students regularly use Git to manage their projects.  The course
generally expands their understanding and usage of Git; most move
from a single branch to using multiple branches, and from pushing
to sending pull requests.  Because many students had learned Git
informally before the class and because the Internet is full of
mediocre advice, some students occasionally employ poor Git practices.
For example, there is a group of students who seem to have learned
to write \texttt{git add .; git commit}.\footnote{That is, add any
new or changed files in or below the current directory and commit
those files to the repository.}  As one might expect, that leads
to some inappropriate materials being added to public repositories.

We discovered one such example indirectly and unpleasantly.  The
instructor for the class received a call from Amazon.com of the
form ``Do you realize that \$5,000 has been charged to your AWS
account in the past two days?''\footnote{During development, the
instructor had allowed the students to set up an AWS account and
to use his credit card.}  It appears that students joining a project
in the second semester had ignored their predecessors instructions
and had managed to put the AWS key in a public repository.  Since
there are people who regularly scrape public repositories for such
keys, it took little time for the now-public key to be found and
used.  Fortunately, Amazon was sympathetic, had also found the
key in the repository, and were willing to waive the charge.

That experience proved instructive, not just for that semester's
students, but for students in subsequent, who were told that any
such experiences in the future would be their financial responsibility.

The initial lessons were not only that there are consequences when
information is inappropriately revealed, but also such experiences
are not uncommon; knowing that Amazon was used to looking for such
issues suggests that.  The more important lessons were about the
technical and nontechnical issues involved in keeping some information
private in a project that is primarily public.

Students needed to figure out technical issues, such as how
to update the \texttt{.gitignore} file to ensure that the key would
not be included, even by someone new to the project who uses `git
add .`, how to configure the system to read that key from an
environment variable, and how to set environment variables in a
platform like Heroku.  But they also needed to decide on practices.
For example, how do you share a private key among group members without
accidentally revealing it to others.  While there are some common practices,
and any organization they joined would have a specific practice, we saw
value in letting students think through this issue on their own and
develop and compare alternative strategies.

\subsection{Revealing personal information}

The lazy Git practice from the prior war story has been at play
in other problematic situations, too.  One group, while developing
an online directory for a local retirement community, accidentally
included photographs of all the residents in their public GitHub
repository.  In addition to the obvious privacy issues, adding a
large number of photographs also significantly increases the download
time for new copies of the repository.

Fortunately, we caught this problem relatively quickly before any
personal information was revealed.\footnote{It helps that the photos
were all in a single zip file.}  But students then had the larger
problem of learning how to ``scrub'' a GitHub repository; it's not
just enough to remove the photographs from the repository, it's
also essential to remove the photographs from the history of the
repository.  Unfortunately, scrubbing repositories ended up being
beyond student capabilities; it required some instructor intervention.

Among other things, this instance highlighted an issue that students
regularly struggle with: The overall structure of the project should
be public so that others can adapt or adopt it.  But some portions
need to be custom for the client.  How do you appropriately separate the
two.  Admittedly, this remains an issue that students struggle with;
too often, they consider the two aspects of the project together,
rather than separately.

\subsection{Changes in management}

We have had a number of instances in which the management of a
partner changes between semesters.  Often, the new manager has a
very different view of what the project should do.  In one case,
the initial project request was for a group text-messaging system
with a Web interface.  We'd been surprised by the original project
requirements, since free software with similar functionality exists.
However, the client felt that they had enough special requirements
that a custom solution was necessary.  When management changed, the
usage model switched significantly.  Now, instead of a Web interface,
the client wanted what was essentially a text-message expander;
they would text to one number and see that text broadcast to a
variety of other numbers.  Students arranged for a phone number,
identified and incorporated SMS receipt libraries, explored security
issues, such as how to limit who could use the service, and designed
a protocol to make it easy for the client to select which list the
message was to be broadcast to.  Then management changed again.  And
the new manager decided that free software did a good enough job.

This was a situation that was particularly difficult to help students
through.  Since management switched, we could not emphasize the idea
that ``Even though they're not using your software, you've helped the
organization grow in understanding.'' After all, each time management 
switched, the knowledge that was developed was lost.   Because similar
free software existed, students did not think that there would be
a benefit to releasing their work as free-and-open-source software.
In the end, we focused on what the students had learned from the project
and on ways they could share that knowledge with future class members
who might need to incorporate SMS in their own work.

\subsection{Uncooperative IT departments}

Unfortunately, the SMS service was not the only instance in which
students ended the project with a sense of disappointment.  We have
had at least two instances in which students finished a project for
a larger non-profit with its own IT department and then discovered that
the IT staff was non-cooperative in getting their own server up and
running.  In one case, the software essentially disappeared after
it was passed on to the client's IT department.  They thanked us,
but they appear not to have had the time or enthusiasm to install
it.  In the second case, the IT staff agreed to provide a server
and to have the students install the software.  But that staff was
slow in responding; they created the server relatively quickly, but
seemed unwilling to open port 80 on that server.  More than two
months of back-and-forth conversations were necessary, which left
the students with insufficient time to release the software.

In both of these cases, we were able to help students focus on the
positives.  While the projects did not get released, they had some
confidence that their software could be used by others.  And, in
both cases, it was clear that the regular meetings with the project
teams had evolved thinking about the projects.

We are still exploring ways to handle this broader issue of
transitioning the project from our students to the organizations.
In most cases, we have found that using a common cloud platform,
such as Heroku, seems to be an approach that serves both the students
and the clients well.

