\section{Course design}

% Janet: Can we focus on the initial design here and leave discussion of changes to the appropriate section? That might eliminate some redundancies.
% Sam: That seems like a good suggestion.  However, I've dropped the changes section, so it may be better to add more here.

In spring 2014, we undertook a complete redesign of the course.
Beyond engaging students in a moderately large project and teaching
software design principles and practices, we had a variety of other
goals.  We wanted to provide students with a more realistic design
experience, one more akin to what they would encounter after college.
We know students are more successful when they are invested in their
projects, so we wanted projects whose benefit they understood.  We
wanted to help students build not only technical skills, but also
soft skills including communication, reflection, and empathy.  
We also felt a strong commitment to
including service learning within our curriculum.  And, to be
honest, we wanted a course that students rated more highly.

\subsection{Primary course characteristics}

These goals led us to a variety of design decisions for the course.

\subsubsection{Community non-profits as project partners}

In determining the kinds of projects we would use, we decided to
focus on local non-profit organizations.  Student engagement with
the community provides an important opportunity.  One of us had
recently worked with a student team to complete a moderate-scale
project for a local organization and it was clear that many similar
organizations in our community also had needs for custom software.
We decided early on that we would focus on non-mission-critical
software; we did not want schedule slippage or poor design to
negatively impact these community partners.\footnote{We use the term ``partner'' to emphasize the importance of
reciprocity in which both students and partner organizations gain
from the relationship.  Using this term rather than ``client'' also
avoids confusion, because most of our partner organizations have
their own clients.} Even with that more limited scope, there is
still a significant need.  As one partner put it, ``There are a lot
of programs that would make our work easier, but that we can't
afford and that would still need customization.  Knowing that we
might eventually get some of that software is a huge benefit.''

At the same time, these projects provide value to our students.  As
is the case in many other courses with significant community service
projects (e.g., \cite{coyle-2005, hfoss-2018}), our students
regularly note that it feels much more rewarding to know that they are
making software that others will use and that will benefit our community,
particularly when compared to projects in other classes, whose audience
appears to be only the professor or classmates.
These community service projects also support
the college's broader mission to ``graduate individuals ... who are
prepared in life and work to use their knowledge and their abilities
to serve the common good.'' \cite{grinnell-mission}

In the four years of the course, we have partnered with a diverse
group of non-profits, including an umbrella ``back office'' that
serves many non-profits in our community, two different food banks,
a local no-kill animal shelter, a retirement community, a
pre-kindergarten program, a retirement home, a sexual assault support
group, and a county-wide service organization.  The diversity of
organizational interests supports a broad spectrum of projects.

\subsubsection{Multi-semester projects}

Inspired by Engineering Projects in Community Service (EPICS) at
Purdue University \cite{coyle-2005}, we decided to adopt multi-semester
projects, which not only allow projects to expand to their natural
scope, but also give students a more realistic experience and help
alleviate some of the end-of-semester stress and frustration.  Since
students often join projects in the middle, they have a realistic
experience of reading and extending legacy code.  By eliminating the
expectation that students finish the project by the end of the
semester, we alleviate the stress and frustration students typically
feel; this model includes neither the traditional end-of-semester
``death march'' nor the dissatisfaction students often feel when 
they must reduce the scope of the project to complete it by the
end of the semester.  Finally, by continuing projects over multiple
semesters, we remove some of the burden of extension and maintenance
from instructors.

Of course, not all projects get finished.  And some that get finished
still don't get adopted.  In those cases, our partner in the 
career-development offices regularly reminds students that the
partner non-profits still received benefit from the \textit{process};
thinking carefully about requirements and developing workflows can
sometimes be as important as the software itself.

\subsubsection{Course structure}

Previously, we offered software design as a single
four-credit course that included both principles and practices
and a project component.  We decoupled the two halves.  That is,
we offered a two-credit course in principles and practices and a
separate two-credit ``practicum'' in which the students apply those
principles and practices to a project with community partners.

Our chief intention was to make it easier for students to repeat
the practicum, either working on the same project for multiple semesters
or joining different projects in different stages of completion.
However, the increased flexibility also made it easier for some students
to put together a course plan. 
%% Particularly
% for students who started the major late or declared two
% majors, it became possible to add an extra course to their program by
% taking principles and practices the first semester and
% the practicum the second semester,
% converting two sixteen-credit semesters to two eighteen-credit
% semesters. 
% Some students who felt particularly
% overwhelmed in a semester or needed to focus on extra-curricular activities
% could drop to a fourteen-credit semester.
The department also started to offer additional two-credit
courses in other areas, making it possible for students to combine
one of the software design half-courses with an elective.

\subsubsection{Alumni mentors}

Students engaging in these types of projects benefit from multiple
kinds of mentoring.  They need advice on design.  They need
assistance to discover which libraries are available and which will
best serve their project.  They need help navigating the complexities
of working with a non-technical partner.  They need
ways to handle the ambiguities inherent in building new software,
from forming appropriate requirements to recovering from unexpected
and inexplicable bugs.

While we have a successful peer mentoring program within the department,
we decided that the students would be better served by alumni mentors
who had significant experience working with Ruby on Rails.
Ruby on Rails is a complex ecosystem and one that evolves
frequently; practitioners who use it daily can draw on
comprehsensive knowledge and understanding than ours.  Alumni mentors
can also give students a sense of how their work
reflected actual industry practice; it's one thing to hear from a
faculty member that a technique or approach is useful, it's another
to hear it from someone who uses it in industry.  The
alumni mentors also provide an opportunity for students to consider
what their life might be like after college.  Many mentors have
dealt with the complexity of non-technical clients and provide a
resource for students struggling with those issues.  By relying on
remote alumni mentors, we would give students the experience of
working with others who are not physically present.  Finally, the
cohort of alumni mentors can often serve as a broader ``panel of
advisors'' to all projects.

This program also builds deeper relationships between
alumni and the college.  Our Office of Alumni Relations reports
that many alumni clamor for more opportunities to support
students.

\subsubsection{Partnering with the career development office}

Because the software design course naturally connects to their
post-college work and because working with a non-technical partner
requires a very different skill set, we partnered with the
career development office to present in class at the beginning of
the semester.

While we formed project teams primarily on the basis of student interest in 
project topics and partners, we
emphasized the importance of diverse skills and perspectives within teams.
For the first two years of the course, a member of the career 
development office joined the class to give students a personal 
inventory, such as Myers-Briggs or Strengths Finder, and to work
with the students to help them explore the meanings of their 
results.\footnote{We have discontinued the use of
these instruments based on discussions with our colleagues in
Psychology who expressed reservations about their validity.  However,
we have found some success in providing
students with a list of characteristics and descriptions of those
characteristics, asking them to pick those that
best describe themselves, and then having them share with their project
team.  That approach provides three useful outcomes:
It helps them understand the kinds of characteristics that are discussed
in the workplace, it encourages them to self-reflect, and it provides
a good starting point for group formation.}
They also helped students understand how different
characteristics affect their groups and the benefits of building
groups with multiple talents.

More importantly, a member of the office helps the students understand
the complexities and subtleties of working in our community.  We
are fortunate that our career development office has a community
outreach coordinator who can speak not only to issues of professionalism,
but also to broader issues of our community.  The discussion touches
on a broad variety of issues, including socio-economic status\footnote{For
example, students are surprised to hear that while the unemployment
rate in town is at about 3\%, over one-third of students in our
school district are eligible for free or reduced lunches.}, network
access\footnote{Students are surprised to hear that for many families,
broadband is not a possibility and cell-phone data is their primary
connection to the Internet.}, biases townspeople have about college
students, and biases students have about the townspeople.

Most importantly, our coordinator helps tease out issues through interactive
discussion, as the following examples illustrate.

\newcommand{\question}[1]{\textsl{#1}}
\newcommand{\answer}[1]{#1}
\newcommand{\followup}[1]{\textsl{#1}}

\question{You want to be professional in your conversations with your client.  Should you contact your client via text messages?}
\answer{No, of course not.}
\followup{You'll find that it depends on the particular partners.  In fact, some of our partners prefer text messages.  It lets them easily group the messages and reply to them on their own timeframe.  In the end, the most important thing is that you understand your client's preferred mode of communication.}

\question{What about clothes?  Should men wear a button-down shirt and a tie to meetings?}
\answer{That seems professional.}
\followup{Once again, it depends on who you are meeting with.  If
you to a meeting with farmers on their farm, and wear dress shoes
and a button-down shirt, they may think that you are setting
yourself above them, that you are not very sensible, or both.
You should usually dress at about the same level as your client,
so make sure to pay attention in your first meeting.  I'm happy
to advise you about what is appropriate for that first meeting.}

\question{You need to meet with your client.  How much lead time is necessary?}
\answer{It depends.}
\followup{It's good to see that you've learned.  Some clients have full
schedules and may want as much as two-weeks of lead time before
you meet.  I'd recommend that you set up your meeting time for the
semester in advance.  You should also agree on mechanisms for
contacting them when you have short followup questions that do not
require an in-person meeting.  Others clients unpredictable schedules
and won't want to schedule more than a few days in advance.  Just make
sure that you learn what is best for your client.}

% Should we insert some more? No, it's already too long.

\subsubsection{Ruby on Rails}

Supporting a wide variety of multi-semester projects suggested that
we should use a single language and platform for all the
projects.
Although students in our program already knew
a variety of languages, including Java, we decided to use Ruby on
Rails as the primary platform.  We saw many advantages to this
platform. Rails embraces a model-view-controller framework that we
consider it important to master. Moreover, Rails is an ``opinionated'' framework
which pushes developers towards consistent coding practices and
design decisions.  When things go right, Rails provides an efficient
and straightforward development environment.  A large number of
modules for common tasks (a.k.a. ``gems'') are available, which
gives students not only a more efficient development pathway, but
also a more realistic development process. Ruby provides a very
different model of object-oriented programming than does Java, which
helps students broaden their understanding of object-oriented
design.\footnote{We found it useful to support this understanding
with a professional text on object-oriented design in Ruby.\cite{poodr}}
There are also many resources for helping students
learn software design with Ruby, including many that are available
for free or at low cost (e.g., \cite{saasbook,rails-tutorial}).

However, the use of Ruby on Rails did not come without other costs.  In
particular, it meant that we had to devote a non-trivial amount of
time during the semester teaching students both Ruby and Rails, time
which delayed the start of their projects.

\subsubsection{Agile project management}

Like many others, we chose to have students manage their project using an 
agile approach.  A few advantages particularly stood out
for this course: User stories are an accessible way to communicate
requirements for non-technical partners, as well as students. 
The product manager role provides a consistent point of contact for partner 
of many organizations.
Many students were already comfortable with pair programming from our 
introductory sequence, and embraced the 
practice of working together in the same room.
Iteration demos hold students accountable on a regular basis.  
We also had students estimate story points, which allowed them to assess 
their own pace and progress by measuring velocity for each iteration.
Iteration retrospectives provide a structured opportunity for students 
to reflect on their experiences and change their practices.  
Since several two-week iterations fit into a 
semester, there are several such opportunities for learning.

One particularly useful concept is time-boxing.
As we all know, projects expand to fill more than available time.
Particularly since the project course was designed as a two-credit 
half-course, we made it clear to students that we were not expecting
a particular product by a particular date; our focus was instead on the
thought and effort they expended on the project.  We encouraged them
to time-box their work: work for at least N hours per week, but no more
than N+M hours.  We also encouraged student to time-box tasks 
and user stories, and to reconsider their planning if not completed in the 
allocated time.  ``Stretch goals'' let students make further progress 
if tasks were completed more quickly than expected.

\subsubsection{An affiliated MOOC/SPOC}

After making many of the design decisions described above, we were
fortunate to discover that Fox and Patterson had released a relevant
textbook \cite{saasbook} and affiliated MOOC (Massive, Open, Online
Course).  
We were particularly excited to find a textbook providing an integrated
approach to learning Ruby on Rails and agile software development, 
including processes for working with a team and with an external client.
We were invited to participate through an affiliated SPOC
(Small, Personalized, Online Course), which provided access to lecture 
videos, programming assignments, and quizzes, 
as well as a community of teachers using the same resources.

The SPOC model gave students the
opportunity to explore online learning, a model
that they will likely need after college.  Through
assignments that asked them to compare their experience reading the
textbook and watching one of the associated recorded lectures, we
were able to help students think about mechanisms of learning that
best serve them.\footnote{Many also appreciated our instructions
to watch the video lecture with no distractions and with a notebook
in hand.} 
