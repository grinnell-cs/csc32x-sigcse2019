\section{Context}

\college\ College is a small (1700 students), rural, highly selective
liberal arts college.  While it is difficult to quantify student
ability, standardized test scores are high: the 75th percentile
scores are 33 (ACT) and 1540 (new SAT composite), and the 25th
percentile scores are 30 and 1370.  Like many institutions, we have
seen our computer science program expand rapidly.  During the time
of this project, we have gone from graduating between twelve and
fifteen CS majors per year to graduating nearly sixty CS majors
each year.

As is often the case at liberal arts colleges, our institution
considers the major a small part of the overall undergraduate
experience.  Policies dictate a particularly lean curriculum for
majors: major requirements may not consist of more than eight courses
plus a few non-departmental requirements.\footnote{Students may
certainly take more than the minimum; nonetheless, requirements
cannot exceed eight courses in the department.} Our current
requirements include a three-course, multi-language, introductory
sequence; systems (either architecture or operating systems);
upper-level algorithm design and analysis; theory of computation;
software design; and one departmental elective.  CS majors must
also take three semesters of mathematics and statistics, including
calculus, discrete mathematics, and a third course of the student's
choice.

Despite these limitations, we provide our students with a strong
and comprehensive undergraduate education, enough so that our
CS curriculum serves as one of the curricular exemplars in the
recent ACM/IEEE Curricular Guidelines \cite{curriculum-2013}.

Nonetheless, the software design curriculum has been particularly
problematic.  As one colleague put it, ``software design often feels
like a bunch of disconnected topics placed together into a single
course, with a large, ambiguous project tied on.''  Throughout the
years, we have offered the course in multiple forms, with different
languages, different emphases, and different textbooks.  In most
cases, students did not take what we hoped out of the course.  The
course also received some of the lower end-of-course ratings in the
department, no matter who taught it.  Even when a model seemed
successful for one semester, subsequent offerings with the same
model showed less success.

