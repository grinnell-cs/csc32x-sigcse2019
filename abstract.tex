\begin{abstract}

For the past four years, we have taught a reimagined software design
course with both typical and atypical components. Projects form the
core of the course: Students work in small teams (4-6 people) to
build non-mission-critical software for local non-profits, building
their engagement with the local community and helping them understand
the broader impact of the work they do. These projects typically
require multiple semesters to complete. Since students most frequently
enroll for the course for one semester, this model gives students
the novel experience of legacy software.  We also provide each team
with an alumni mentor who helps them navigate not only technical
problems but also the challenges of working with a real-world,
non-technical client.

These aspects of the course also develop our students' soft skills.
They learn to work with a team, to communicate with non-technical
clients, to work with remote collaborators (or mentors), and to
think ahead to the successors who have to take on the project in
the next semester. As we tell our students, these skills are often
as crucial as their technical skills.

In this paper, we report on the design of the course and describe
many of the challenges associated with this model (e.g., projects
that inadvertently reveal confidential information or keys to costly
services, clients who switch management and, therefore, expectations,
and projects that become obsolete).  We also provide suggestions
for those who might want to adopt a similar approach, including
strategies for recruiting and managing partners and alumni.

\end{abstract}
