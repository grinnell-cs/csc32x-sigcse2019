\begin{abstract}

For the past four years, we have taught a reimagined software design
course with both typical and atypical components. Projects form the
core of the course: Students work in small teams (4-6 people) to
build non-mission-critical software for local non-profits. These
projects typically require multiple semesters to complete. Since
students usually enroll for the course for one semester, this model
gives students the novel experience of legacy software. We also
provide each team with an alumni mentor who helps them navigate not
only technical problems, but also the challenges of working with a
real-world, non-technical client.

These aspects of the course also help us develop our students' "soft
skills". They learn to work with a team, to communicate with
non-technical clients, to work with remote collaborators (or mentors),
and to think ahead to the successors who have to take on the project
in the next semester. As we tell our students, in many ways, these
skills are more important than the technical skills they develop
in the course.

In this paper, we report in detail about the design of the course
and describe many of the challenges associated with this model,
including projects that inadvertently reveal confidential information
or keys to costly services, clients who switch management and,
therefore, expectations, and projects that become obsolete before
they are finished. We also provide suggestions for those who might
want to adopt a similar approach, such as strategies for recruiting
and managing partners and alumni.

\end{abstract}
