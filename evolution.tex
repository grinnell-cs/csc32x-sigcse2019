\section{Evolving the course}

The course has evolved somewhat through the years from its original
design.  In this section, we describe some of the important changes
and explain our rationale for those changes.

\textit{Introducing Rails}.  For the first few iterations of the
course, we introducted Rails through the Rotten Potatoes exercises
from the SaaS course.  That strategy provided some benefits: It was
a moderate scale program that explored a variety of issues of Rails
design through a series of exercises.  The auto-grader also gave
students more experience in and understanding of test-driven
design.\footnote{It was also a very different experience of grading
for students used to individual hand grading of their assignments.}
There is enough infrastructure that students could progress to the
next assignment even after struggling on the previous assignment
by relying on the starter code for each assignment.  The example
is also integrated into the textbook \cite{saasbook}.

At the same time, we found some significant drawbacks.  Too often,
when they encountered a problem, a Web search revealed not just
advice, but an answer akin to ``If you have \textit{this error} in
your Rotten Potatoes assignment, insert \textit{this code}.''  The
availability of such answers undermined our goal of having students
navigate the complexities of unknown errors.  Students also found
themselves frustrated with the easy answers; they wanted broad
advice of the form ``consider the following potential approaches'',
not detailed advice that they were expected to plug in.  They clearly
cared more about learning than about simply finishing the assignment.
Students also expressed some frustration at working in a model in
which they were expected to change a few things in an existing
framework; while that's a realistic experience, given that they
would be getting that experience in the broader project getting a
``from scratch'' development experience seemed more important.
We also found that it took too much of the semester for students 
to get through the assignments, delaying their work on the broader
project.

As a replacement, we have used Michael Hartl's \textit{Ruby Tutorial}
\cite{ruby-tutorial}.  Not only does it guide students step-by-step
through the design and implementation of a realistic Ruby application,
it is available for free online.  However, it quickly became clear
that we could not just present the exercises by themselves; while the 
best students would reflect on each instruction, others would do nothing
other than type each instruction in sequence.  We solved this issue by
adding a regular reading journal assignment that leads students into the
deeper reflections.

\textit{Personality inventories}.  As we noted earlier, we employed
personality inventories, such as Myers-Briggs and Strengths Finder,
in early iterations of the course.  However, conversations with
members of the psychology faculty about these inventories revealed
that they had concerns about the use and reliability of such
inventories.  Students also expressed some frustration with the
form of some inventories.  We have found some success in providing
students with a list of characteristics and descriptions of those
characteristics and then asking them to pick the characteristics that
best describe themselves.

\textit{Offering times}.  In the first iteration, the principles
and practices course met for two one-hour slots and the practicum
met once per week for three hours.  Through the years, a variety
of issues have led us to make significant changes to the offering.
For example, some students found the sustained three-hour practicum
overwhelming and many students reported that they were having
difficulty finding time to meet with their project team outside of
class.  We also found that we needed to spend more time on the
principles and practices at the start of the semester and more time
on the practicum at the end of the semester.  We settled on a model
in which the principles and practices course met for three one-hour
slots per week for the first half of the semester with an expectation
of about nine-hours per week of out of class per week\footnote{A
two-credit class is expected to involve about ninety hours of student
work throughout the semester} and the practicum meeting for three
one-hour slots per week for the first half of the semester and then
took over the slots of the the principles and practices course for
the second half of the semester, meeting in three two-hour slots
per week.

\textit{Re-coupling the courses}.  One of the original characteristics
of this project was the decoupling of the principles and practices component
of the course from the associated practicum.  While we saw many
benefits to that approach, we encountered a number of obstacles.
While most students took both courses in the same semester, a few
took them in separate semesters.  The separate-semester students
experienced some difficulties: they forgot some of what they had
learned in the principles and practices course, particularly if
there was an intervening semester; they were ready to begin the
larger project before their colleagues, so the first few weeks of
the semester proved less useful; and they felt less connected to
the class cohort.  We also encountered some administrative difficulties
in the split format, particularly in terms of crediting the teaching
effort.  Hence, we have now re-coupled the two courses into a single
four-credit course.  We have also added a two-credit course to accommodate 
students who want to return to a project for a second 
semester.\footnote{Getting the Registrar to permit two courses in the same
room at the same time was an interesting challenge.}  

\textit{Deeper coverage of object-oriented programming}.  This
course was not the only one evolving over this time period.  At the
same time that we were developing and evolving our software design
course, the instructors in charge of our course in data structures,
algorithms, and object-oriented design were also making signficant
changes.  In particular, while they continued to introduce the key
ideas of encapsulation, inheritance, and polymorphism (both subtype
and parametric), they de-emphasized other aspects of object-oriented
design.  We also started getting students earlier in their careers;
what had traditionally been a course that students took in their
junior or senior years became a course that some students were
taking as early as spring of their sophomore year.  Both issues led
to students entering the class with a less-deep understanding of
OOD practices.  We addressed this issue by incorporating an additional
textbook on object-oriented design \cite{poodr} and devoting one day
per week during the first half of the semester on issues raised by that
book.  Once again, reading journals proved essential in ensuring that
students read at more than surface level and considered key issues.

