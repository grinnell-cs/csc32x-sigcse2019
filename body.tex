\section{Introduction}

Many forms of software design.  (Perhaps a small it review.)
Fox suggests a model based on software as a service \cite{fox-2014}.
Open-source, particularly with a focus on humanitarian free and
open-source software (HFOSS) \cite{hfoss-2018}.  Formal methods.
Projects or not.  Yadda yadda yadda.

The Software Design curriculum has been a particularly problematic issue
at our institution.  ... As one colleague put it, ``Software Design often
feels like a bunch of disconnected topics placed together into a single
course, with a large, ambiguous project tied on.''  Throughout the years,
the course was offered in multiple forms, with different languages,
different emphases, and different textbooks.  But the general experience
was that students did not take what we hoped out of the course and that
the course received some of the lower end-of-course ratings in the department,
no matter who taught it.

Since most commercial software is not finished on time \cite{fox-2014},
is it reasonable that we expect students to finish their own projects
in a semester and feel proud of what they've done; finishing usually
requires compromises.

Paragraph summarizing the key points of the project.

\section{Course design}

Goals: More realisitic software design experience.  Projects that
students feel invested in.  Build both soft and technical skills.
And, to be honest, a software design that courses

\subsection{Course components}

\subsubsection{Multi-semester projects}

Why?  Realistic experience.

\subsubsection{Community non-profits as project partners}

Value to them.  Value to us.  Value to the students.

\subsubsection{Ruby on Rails}

\subsubsection{Alumni mentors}

\subsubsection{Partnering with the career-development office}

One aspect: Giving and helping them understand personal characteristic
inventories, such as Myers-Briggs or Strengths Finder.

A second aspect: Helping them develop and understand how to work with
community members.  We are fortunate that CDO has a community outreach
coordinator who can not only speak to issues of professionalism, but
also to broader issues of our community.  SES issues.  Biases on both
sides.  Here are some examples from a typical class session.

\newcommand{\question}[1]{\textbf{\textsl{#1}}}
\newcommand{\answer}[1]{#1}
\newcommand{\followup}[1]{#1}

\question{You want to be professional in your conversations with yoru client.  Should you contact your client via text messages?}
\answer{No, of course not.}
\followup{Actually, it depends on the particular partners.  You'll find that some of our partners actually prefer text messages.  It lets them easily group the messages and reply to them on their own timeframe.  In the end, the most important thing is that you understand that clients' preferred mode of communication.}

\question{What about clothes?  Should men wear a button-down shirt and a tie to meetings?}
\answer{That seems professional.}
\followup{Once again, it depends on who you are meeting with.  If you to a meeting with farmers on their farm, and wear dress shoes and a button-down shirt and tie, they may think that you are setting yourself above them, that you are not very sensible, or both.}

\subsubsection{An affiliated MOOC/SPOC}

Started with SaaSbook \cite{saasbook}.  But the core SaaSbook project
does not fit well into the "there are no easy answers" model;
solutions to every problem are somewhere on StackOverflow.  Students
also had mixed experience with participating in a MOOC.

\subsection{Course Structure}

\section{Sample Projects}

\section{War stories and corresponding lessons}

\subsection{Revealing an AWS key}

Unfortunately, some students have learned the poor Git practice of
\texttt{git add .; git commit}.\footnote{That is, add any new or
changed files in or below the current directory and commit those files
to the repository.}  That means that

Replace key.

Learn how to avoid putting the key in the repo.

Learn how to configure the system to grab the key from the environment
variables.

Determine a strategy for sharing this information with other group members.
Rather than specifying a particular practice, as is likely to happen in
most professional environments, we 

\subsection{Revealing personal information}

The lazy Git practice from the prior war stories has been at play
in other problematic situations, too.  One group, while developing
an online directory for a local retirement community, accidentally
included photographs of all the residents in their public GitHub
repository.  In addition to the obvious privacy issues, adding a
large number of photographs also significantly increases the download
time for new copies of the repository.

Fortunately, we caught this problem relatively quickly.  But students
then had the larger problem of learning how to ``scrub'' a GitHub
repository; it's not just enough to remove the photographs from the
repository, it's also essential to remove the photographs from the
history of the repository.  Unfortunately, scrubbing repositories
ended up being beyond student capabilities; it required some
instructor intervention.

Lesson: Distinguish between the general project (a directory system) and
the details of the particular project.  Remains an issue that students
struggle with.

\subsection{Changes in management}

\subsection{Uncooperative IT staffs}

\subsection{Disappearing projects}

\section{Outcomes}

Soft skills.

An understanding of the value of what you do.

We hope: Sense of self efficacy: "Hey, I can throw together a decent
database-backed Web site in a week, at least if everything goes right."

We hope: Broader "learning how to learn".  In a situation in 

As one student puts it, ``In this course, more than most, what you get
out of this course depends directly on what you put into the course.''
That is,

\section{Notes on the Evolution of the Course}

\section{Recommendations for Others}

\subsection{Group formation}

There's a rich literature.  We've used a variety of approaches.  Using
Myers-Briggs or StengthsFinder styles ended up complicating the process.
Asking students to list ``anti-preferences'' rather than preferences
often led to some students being ostracised based on reputation rather
than for valid reasons.\footnote{We do make it clear to students that
they can make special requests based on past history with another student;
such requests are rare.}

Currently ...

\subsection{Identifying mentors}

We used LinkedIn to identify potential alumni mentors.  Response rate
to invitations was high.  Depending on the institutional context, other
approaches might also work: Departmental alumni database, institutional
or departmental alumni email list, etc.

\subsection{Partnering with career development}

We have found it very productive to partner with our career development
office.  Although we are no longer using them to give and talk through
skills assessments, the training they give students before meeting with
community partners and the guidance that they give throughout the semester
are invaluable.  For those reasons, we highly recommend that you partner
with your own office.

If you are unable to find an appropriate partner, it is possible to run
your own ``preparing to meet with your community partner''.  Issues to
focus on include ...


\subsection{Identifying community partners}

Fortunate to have an outreach coordinator.  Makes sure that both sides
don't overcommit.  Makes sure community partner understands the limitations
of working with students.

Even if you don't have an outreach coordinator, it may be helpful to
bring a third person to meetings to provide some of this kind of context.

\subsection{Time boxing}

As all of us know, projects expand to fill more than available time.
Particularly since the original model of the course was a two-credit
project course, we made it clear to students that we were not 

\subsection{Contextualizing the course}

At least at our institution, 

\section{Conclusion}

One two three four five six seven eight nine ten.  This paragraph exists
mostly to make sure that I have more than two pages and can see what
happens on the third page.
