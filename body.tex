\section{Introduction}

While software design and software engineering form a core part of
the undergraduate computer science curriculum, perspectives on what
form that course or courses take vary significantly.  Some courses
focus on formal methods (e.g., \cite{liu-2009}, \cite{gracia-2014}).
Some consider topics more broadly, such as through design patterns,
comparisons of plan-and-document and agile methodologies
\cite{gestwicki-2018}.  Some excite students through the use of
video games (e.g., \cite{wolz-2007}.  Some courses are now adopting
an emphasisa on open source, such as the Posse group's work on
humanitarian free and open-source software (HFOSS) \cite{hfoss-2018}.

Programs also vary on the role of projects.  Some curricula place
a small project in the software design or software engineering
course and then follow that with a capstone project.  For others,
the project is the center of the software design course, with
everything else done in support of it.

Since most commercial software is not finished on time \cite{fox-2014},
is it reasonable that we expect students to finish their own projects
in a semester and feel proud of what they've done; finishing usually
requires compromises.

In this paper, we explore the software design model that we have
meployed for the past four years.  That model, while including many
common characteristics of traditional software design courses, also
introduces some imporant innovations.  These innovations include
(a) multi-semester projects that students often join in the middle,
(b) community non-profit organizations serve as the primary client
base for the projects, (c) separate half-courses, one that introduces
the principles and practices necessary for the design of a medium-scale
project, the other of which focuses primarily on the projects, (d)
alumni project mentors, who help guide project teams, (e) a partnership
with our career-development office, (f) an emphasis on Web-based
software with a common platform, and (g) in many offerings, the use
of a MOOC/SPOC to ground learning.

These characteristics fit together in a way that not only provides
most students with a strong background in software design, but also
helps build a variety of soft skills that are likely to have at
least as much benefit for our students.

In section 2 of this paper, we describe the institutional and
departmental context for the curriculum.  In section 3, we explore
the goals and design of the new software design component and
describe some of the projects that students have develped.  In
section 4, we report on selected ``war stories'' that others may
find instructive.  In section 5, we describe some outcomes of the
course.  In section 6, we consider some of the changes of the course
over four years..  Finally, in section 7, we provide recommendations
for others intersted in adapting some or all of these practices.

\section{Context}

\college College is a small (1700 students), rural, highly selective
libreral arts college.  Students have generally shown academic
success; For example, the 75th percentile scores are 33 (ACT) and
1540 (new SAT composite); the 25th percentile scores are 30 and
1370.  Like many institutions, we have seen our program expand
rapidly.  During the time of this project, we have gone from
graduating between twelve and fifteen CS majors per year to graduating
nearly sixty CS majors each year.

Like many liberal arts colleges, \college considers the major a
small part of the overall undergraduate experience.  \college expects
a relatively lean curricular for majors; majors requirements may
not consist of more than eight courses plus a few non-departmental
requirements.\footnote{Students can certainly do more than the
minimal requirements.  But the requirements cannot exceed eight
courses in the department.} Our current departmental curricular
requirements include of a three-course, multi-language, introductory
sequence; systems (either architecture or operating systems);
upper-level algorithm design and analysis; theory of computation;
software design; and one departmental elective.  CS majors must
also take three semesters of mathematics and statatistics, including
Calculus, discrete mathematics, and a third course of the student's
choice.  

In spite of these limitations, we provide our students with a strong
and comprehensive undergraduate aducation, enough so that \college's
CS curriculum serves as one of the curricular exemplars in the
recent ACM/IEEE Curricular Guidelines \cite{curriclum-2013}.

Nonetheless, the Software Design curriculum has been particularly
problematic.  As one colleague put it, ``Software Design often feels
like a bunch of disconnected topics placed together into a single
course, with a large, ambiguous project tied on.''  Throughout the
years, the course has been offered in multiple forms, with different
languages, different emphases, and different textbooks.  In most
cases, the experience was that students did not take what we hoped
out of the course and that the course received some of the lower
end-of-course ratings in the department, no matter who taught it.
Even when a model seemed successful for one semester, subsequent
offerings in the same model showed less success.

\section{Course design}

Because of these issues, in spring 2013 we undertook a complete
redesign of the course.  In developing the new version, we kept a
variety of core goals in mind.  The software design course serves
two major roles in the curriculum: It introduces students to the
skills and principles that allow them to contribute to medium and
large scale projects and it gives them the experience of working
on a larger project than they typically encounter in other courses.

Within that framework, we had a variety of other goals.  We wanted
to provide students with a more realistic design experience, one more
akin to what they would encounter after college.  We know students
are more successful when they are invested in their projects, so we
wanted projects whose benefit they understood.  We wanted to help
students build not only technical skills, but also so-called ``soft
skills''.  

And, to be honest, we wanted a course that students rated more highly.

\subsection{Primary course characteristics}

These goals led us to a variety of design decisions for the course.

\subsubsection{Multi-semester projects}

We had traditionally focused on projects that students could complete
in one semester.  Full-semester projects are clearly more substantial
than the typical class project.  However, full-semester projects
in a four-credit class are still substantially smaller than the
projects students encounter after college; the limitation that "it
must be completed in a semester" seems, well, artificially limiting.
In addition, most student projects, like most real-world projects,
take more time than anticipated.  For students, this often leads
to an end of semester in which they struggle to finish something
and, whatever they finish, it often feels like less than they wanted.

We decided that adopting multi-semester projects would not only
give students a more realistic experience but would also help
alleviate some of the end-of-semester stress and frustation.  Since
students often join projects in the middle, the have the more common
experience of needing to understand and find ways to contribute to
an existing code base.  Since there is not an expectation that
students finish the project by the end of the semester, we alleviate
some of the stress students typically feel.

\subsubsection{Community non-profits as project partners}

In determining the kinds of projects we would use, we decided to
focus on local non-profit institutions.  One of us had recently
completed a moderate-scale project for a local organization and it
was clear that many similar organizations in our community also had
needs for custom software.  We decided early on that we would focus
on non-mission-critical software; we did not want project slippage
or failed design to negatively impact these community partners.

But there's still a significant need for non-mission-critical software.
As one partner put it, ``There are a lot of programs that would make
our work easier, but that we can't afford and that would still need
customization.  Knowing that we might eventually get some of that
software is a huge benefit.''  That knowledge also appears to make the
partners patient when a project takes longer than even we expect, or
when we have to put a project on hold for unexpected reasons. So the
projects provide real value to our partners.

At the same time, these projects provide value to our students.  They
regularly note that it feels much more rewarding to know that they are
making software that others will use and that will benefit our community,
particularly when compared to the projects that they do for other classes,
which they think of as primarily for themselves or their professor.

Of course, not all projects get finished.  And some that get finished
still don't get adopted.  In those cases, our partner in the 
career-development offices regularly reminds students that the
partner non-profits still received benefit from the \textit{process};
thinking carefully about requirements and developing workflows can
sometimes be as important as the software itself.

\subsubsection{Ruby on Rails}

Supporting a wide variety of multi-semester projects suggested that
we should use a single language and platform for all the
projects.\footnote{We did try adding a second platform in two
semesters; that ended up making the work of both faculty and students
much more complex.}  Although students in our program already knew
a variety of languages, including Java, we decided to use Ruby on
Rails as the primary platform.  We saw many advantages to this
platform: Rails embraces a model-view-controller framework that we
consider it important to master; when things go right, Ruby on Rails
provides an efficient and straightforward development environment;
a large number of modules for common tasks (aka ``gems'') are
available, which gives students not only a more efficient development
pathway, but also a more realistic development process; Ruby
provides a very different model of object-oriented programming than
does Java, which helped students broaden their understanding of
object-oriented design; and there are a wide array of resources for
helping students learn software design with Ruby, including many
that are available for free or at low cost (e.g.,
\cite{saasbook,railstutorial}).

However, the use of Ruby on Rails did not come without other costs.  In
particular, it meant that we had to devote a non-trivial amount of
time during the semester teaching students both Ruby and Rails, time
which delayed the potential start of their projects.

\subsubsection{Alumni mentors}

Students engaging in a large project benefit from a variety of kinds
of mentoring.  They need advice on design.  They need advice on
which libraries (gems) are available and which will best serve their
project.  They need help navigating the complexities of working
with a non-technical partner.  In many cases, they need ways to
handle the ambiguities inherent in building new software, from
forming appropriate requirements to handling unexpected and
unexplicable bugs.

While we have a successful peer mentoring program within the department,
we decided that the students would be better served by alumni mentors
who had significant experience developing software in Ruby on Rails.
Ruby on Rails is a complex enough ecosystem, and one that evolves
frequently enough, that it was unlikely that either we or the class
mentor would have as comprehensive knowledge as someone who uses
it daily.  It was clear that alumni mentors would give students a
broader sense of how their class practice reflected actual industry
practice; it's one thing to hear from a faculty member that a
technique or approach is useful, it's another to hear it from someone
who uses the technique or approach in their regular work.  The
alumni mentors also provide an opportunity for students to consider
what their life might be like after college.  Many mentors have
dealt with the complexity of non-technical clients and Brovided a
resource for students struggling with those issues.  By relying on
remote alumni mentors, we would give students the experience of
working with others who are not physically present.  Finally, the
cohort of alumni mentors can often serve as a broader ``panel of
advisors'' to all projects.

Of course, having alumni mentors also helps build connections between
the alumni and the college.  We hear regularly from our Office of
Alumni Relations that many alumni clamor for more opportunities to
help support students.

\subsubsection{Partnering with the career-development office}

Because the software design course naturally connects to their
post-college work and because working with a non-technical partner
requires a very different skillset, we partnered with the 
career-development office to present in class at the beginning
fo the semester.
of the semester. 

For the first two years of the course, a member of that office
joined the class to give the students a personal characteristic
inventory, such as Myers-Briggs or Strengths Finder, and to work
with the students to help them explore the meanings of their results.
They also helped the students understand how the different
characteristics affected their groups and then benefits of building
groups with multiple talents.  Although we discontinued the use of
these instruments based on some discussions with our colleages in
the Psychology department, we continue to use some of the ideas in
helping students think about the diversity of skills necessary for
a successful group.

More importantly, a member of the office helps the students understand
the complexities and subtleties of working in our community.  We
are fortunate that our career development office has a community
outreach coordinator who can speak not only to issues of professionalism,
but also to broader issues of our community.  The discussion touches
on a broad variety of issues, including socio-economic status\footnote{For
example, students are surprised to hear that while the unemployment
rate in town is at about 3%, over one-third of students in our
school district are eligible for free or reduced lunches.}, network
access\footnote{Students are surprised to hear that for many families,
broadband is not a possibiilty and cell-phone data is their primary
connection to the Internet}, biases townspeople have about college
students, and biases the students have about the townspeople.

Most importantly, our coordinator helps tease out issues through interactive
discussion.  Here are some exmaples from a typical class session.

\newcommand{\question}[1]{\textsl{#1}}
\newcommand{\answer}[1]{#1}
\newcommand{\followup}[1]{\textsl{#1}}

\question{You want to be professional in your conversations with yoru client.  Should you contact your client via text messages?}
\answer{No, of course not.}
\followup{Actually, it depends on the particular partners.  You'll find that some of our partners actually prefer text messages.  It lets them easily group the messages and reply to them on their own timeframe.  In the end, the most important thing is that you understand that clients' preferred mode of communication.}

\question{What about clothes?  Should men wear a button-down shirt and a tie to meetings?}
\answer{That seems professional.}
\followup{Once again, it depends on who you are meeting with.  If you to a meeting with farmers on their farm, and wear dress shoes and a button-down shirt and tie, they may think that you are setting yourself above them, that you are not very sensible, or both.}

% Should we insert some more?

\subsubsection{Course structure}

Traditionally, the software design course has been offered as a single
four-credit course that included both the principles and practices and
the project component.  We decoupled the two halves.  That is, we 
offered a paired two-credit course in principles and practices and
a separate two-credit ``practicum'' in which the students apply
those principles and practices to a project with community partners.

The primary intention was to make it easier for students to repeat
the practicum, either working on a project for multiple semesters
or joining different projects in different stages of compleition.
However, there were a host of other potential benefits.  For some
students, it made it easier to put together a plan; particularly
for students who started the major late or who were completing two
majors, it became possible to add an extra course to their plan by
converting two sixteen-credit semesters to two eighteen-credit
semesters (taking principles and practices the first semester and
the practicum the second semester).  Some students who felt particularly
overwhelmed in a semester or needed to focus on extra-curricular activities
could drop to a fourteen-credit semester.

At the same time, we also started to offer additional two-credit
courses in other areas, making it possible for students to combine
one of the software design half-courses with an elective.

\subsubsection{An affiliated MOOC/SPOC}

After making most of the design decisions described above, we were
fortunate to discover that Fox and Patterson had released not only
a textbook \cite{saasbook} but also an affiliated MOOC (Massive,
Open, Online Course).  Through some negotiations, we were able to
get an affiliated SPOC (Small, Personalized, Online Course) craeated.
In addition to providing us with a wealth of resources for the
course, this SPOC model gave students the opportunity to explore
online learn, one of the models of learning that they will likely
need to employ after college.  Through assignments that asked them
to compare their experience reading the textbook and watching one
of the associated recorded lectures, we were able to help students
think about mechanisms of learning that best serve them.\footnote{Many also
appreciated our instructions to watch the video lecture with no distractions
and with a notebook in hand.}

\subsection{Projects}

In the four years of the course, we have partnered with a wide
variety of non-profits, including an umbrella ``back office'' that
serves many nonprofits in our community, two different food banks,
a local no-kill animal shelter, a retirement community, a
pre-kindergarten program, a retirement home, a sexual assualt support
group, and a county-wide service organization.  These organizations
have presented us with a wide spectrum of projects.

In building an online directory for a retirement community, students
found themselves exploring issues of accessibility (e.g., elderly
have different needs than more traditional computer users), privacy,
control, and more.

In developing grant management software for the umbrella organization,
students learned about the ad-hoc solutions that organizations may employ
when software is too expensive and considered ways to manage files and
maintain privacy in cloud storage.

% Yeah, I probably need to add a few more.

\section{War stories}

Of course, not everything in the course went smoothly.  In this section,
we describe some of the more significant problems that arose throughout
the years and suggest some lessons that each reveals.

\subsection{Revealing an AWS key}

Students regularly use Git to manage their projects.  The course
generally expands their understanding and usage of Git; most move
from a single branch to using multiple branchs, and from pushing
to sending pull requests.  Because many students had learned Git
informally before the class and because the Internet is full of
mediocre advice, some students ocasionally employ poor Git practices.
For example, there is a group of students who seem to have learned
to write \texttt{git add .; git commit}.\footnote{That is, add any
new or changed files in or below the current directory and commit
those files to the repository.}  As one might expect, that leads
to some inappropriate materials being added to public repositories.

We discovered one such example indirectly and unpleasantly.  The
instructor for the class received a call from Amazon.com of the
form ``Do you realize that \$5,000 has been charged to your AWS
account in the past two days?''\footnote{During development, the
instructor had allowed the students to set up an AWS account and
to use his credit card.}  It appears that students joining a project
in the second semester had ignored their predecessors instructions
and had managed to put the AWS key in a public respository.  Since
there are people who regularly scrape public repositories for such
keys, it took little time for the now-public key to be found and
used.  Fortunately, Amazon was sympathetic, had also found the
key in the repository, and were willing to waive the charge.

That experience proved instructive, not just for that semester's
students, but for students in subsequent, who were told that any
such experiences in the future would be their financial responsibility.

The initial lessons were not only that there are consequences when
information is inappropriately revealed, but also such experiences
are not uncommon; knowing that Amazon was used to looking for such
issues suggests that.  The more important lessons were about the
technical and nontechnical issues involved in keeping some information
private in a project that is primarily public.

Students needed to figure out technical issues, such as how
to update the \texttt{.gitignore} file to ensure that the key would
not be included, even by someone new to the project who uses `git
add .`, how to configure the system to read that key from an
environment variable, and how to set environment variables in a
platform like Heroku.  But they also needed to decide on practices.
For example, how do you share a private key among group members without
accidentally revealing it to others.  While there are some common practices,
and any organziation they joined would have a specific practice, we saw
value in letting students think through this issue on their own and
develop and compare alternative strategies.

\subsection{Revealing personal information}

The lazy Git practice from the prior war story has been at play
in other problematic situations, too.  One group, while developing
an online directory for a local retirement community, accidentally
included photographs of all the residents in their public GitHub
repository.  In addition to the obvious privacy issues, adding a
large number of photographs also significantly increasds the download
time for new copies of the repository.

Fortunately, we caught this problem relatively quickly before any
personal information was revealed.\footnote{It helps that the photos
were all in a single zip file.}  But students then had the larger
problem of learning how to ``scrub'' a GitHub repository; it's not
just enough to remove the photographs from the repository, it's
also essential to remove the photographs from the history of the
repository.  Unfortunately, scrubbing repositories ended up being
beyond student capabilities; it required some instructor intervention.

Among other things, this instance highlighted an issue that students
regularly struggle with: The overall structure of the project should
be public so that others can adapt or adopt it.  But some portions
need to be custom for the client.  How do you appropriately separate the
two.  Admittedly, this remains an issue that students struggle with;
too often, they consider the two aspects of the project together,
rather than separately.

\subsection{Changes in management}

We have had a number of instances in which the management of a
partner changes between semesters.  Often, the new manager has a
very different view of what the project should do.  In one case,
the initial project request was for a group text-messaging system
with a Web interace.  We'd been surprised by the original project
requirements, since free software with similar functionality exists.
However, the client felt that they had enough special requirements
that a custom solution was necessary.  When management changed, the
usage model switched significantly.  Now, instead of a Web interface,
the client wanted what was essentially a text-message expander;
they would text to one number and see that text broadcast to a
variety of other numbers.  Students arranged for a phone number,
identified and incorporated SMS receipt libraries, explored security
issues, such as how to limit who could use the service, and designed
a protocol to make it easy for the client to select which list the
message was to be broadcast to.  Then management changed again.  And
the new manager decided that free software did a good enough job.

This was a situation that was particularly difficult to help students
through.  Since management switched, we could not emphasize the idea
that ``Even though they're not using your software, you've helped the
organization grow in understanding.'' After all, each time management 
switched, the knowledge that was developed was lost.   Because similar
free software existed, students did not think that there would be
a benefit to releasing their work as free-and-open-source software.
In the end, we focused on what the students had learned from the project
and on ways they could share that knowledge with future class members
who might need to incorporate SMS in their own work.

\subsection{Uncooperative IT staffs}

Unfortunately, the SMS service was not the only instance in which
students ended the project with a sense of disappointment.  We have
had at least two instances in which students finished a project for
a larger non-profit with its own IT staff and then discovered that
the IT staff was non-cooperative in getting their own server up and
running.  In one case, the software essentially disappeared after
it was passed on to the client's IT department.  They thanked us,
but they appear not to have had the time or enthusiasm to install
it.  In the second case, the IT staff agreed to provide a server
and to have the students install the software.  But that staff was
slow in responding; they created the server relatively quickly, but
seemed unwilling to open port 80 on that server.  More than two
months of back-and-forth conversations were necessary, which left
the students with insufficient time to release the software.

In both of these cases, we were able to help students focus on the
positives.  While the projects did not get released, they had some
confidence that their software could be used by others.  And, in
both cases, it was clear that the regular meetings with the project
teams had evolved thinking about the projects.

\section{Outcomes}

Soft skills.

An understanding of the value of what you do.

We hope: Sense of self efficacy: "Hey, I can throw together a decent
database-backed Web site in a week, at least if everything goes right."

We hope: Broader "learning how to learn".  Regularly face situations
in which they can't just find an answer in a textbook.  Also many
situations in which they can't just find the answer on StackOverflow.
So a lot of practice in finding the right way to do a Web search,
how to trace issues through a program, interpret waht they find online,
and more.

Tolerance of ambiguity.  Used to clearly specified assignments.  These
are open-ended.

Unfortunately, we have not seen as many positive effects of
end-of-course evaluations as we had hoped.  While some students
consider the course the center of their education, others seem to
find it pointless.  As one student puts it, ``In this course, more
than most, what you get out of this course depends directly on what
you put into the course.''

\section{Notes on the Evolution of the Course}

\section{Recommendations for Others}

\subsection{Group formation}

There's a rich literature.  We've used a variety of approaches.  Using
Myers-Briggs or StengthsFinder styles ended up complicating the process.
Asking students to list ``anti-preferences'' rather than preferences
often led to some students being ostracised based on reputation rather
than for valid reasons.\footnote{We do make it clear to students that
they can make special requests based on past history with another student;
such requests are rare.}

Currently ...

\subsection{Identifying mentors}

We used LinkedIn to identify potential alumni mentors.  Response rate
to invitations was high.  Depending on the institutional context, other
approaches might also work: Departmental alumni database, institutional
or departmental alumni email list, etc.

\subsection{Partnering with career development}

We have found it very productive to partner with our career development
office.  Although we are no longer using them to give and talk through
skills assessments, the training they give students before meeting with
community partners and the guidance that they give throughout the semester
are invaluable.  For those reasons, we highly recommend that you partner
with your own office.

If you are unable to find an appropriate partner, it is possible to run
your own ``preparing to meet with your community partner''.  Issues to
focus on include ...


\subsection{Identifying community partners}

Fortunate to have an outreach coordinator.  Makes sure that both sides
don't overcommit.  Makes sure community partner understands the limitations
of working with students.

Even if you don't have an outreach coordinator, it may be helpful to
bring a third person to meetings to provide some of this kind of context.

\subsection{Time boxing}

As all of us know, projects expand to fill more than available time.
Particularly since the original model of the course was a two-credit
project course, we made it clear to students that we were not 

\subsection{Contextualizing the course}

At least at our institution, 

\section{Conclusion}

One two three four five six seven eight nine ten.  This paragraph exists
mostly to make sure that I have more than two pages and can see what
happens on the third page.
