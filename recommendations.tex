\section{Conclusion and recommendations}

By using a variety of traditional (e.g., real-world projects, agile
practices) and novel components (e.g., alumni mentors, multi-semester
projects, collaboration with the career development office), this
course gives students more realistic experiences in software design.
These experiences develop not only technical skills but the less
quantifiable soft skills that are essential to their future success.

While we find the combination of course design components particularly
successful, others may find it useful to adapt just a few.  Recommendations
associated with some of the more novel aspects of the course follow.

\subsection{Identifying alumni mentors}

We have found LinkedIn to be a particularly useful tool for identifying
alumni mentors.  A search for people who attended the institution and
who had experience with Ruby on Rails produced a small, but sufficiently
large, list.  Discussions with other faculty in the department helped
us choose who to prioritize.  The response rate to invitations was high.

Other approaches are possible.  One might search a departmental
alumni database, if one is available.  One might broadcast a message
to a mailing list of departmental alumni, although that approach
might have the disadvantage of some alumni feeling disheartened
that they volunteered and were not selected.  They might be invited
to serve as readers of the projects or to provide backup
advice.\footnote{For example, please monitor this Slack channel for
general questions on Ruby, Rails, and object-oriented design.}

It is also worth considering what to do across semesters: Do mentors
continue from semester to semester, do you recruit a new cohort
each semester, or do you rotate a few in and out each semester?  We
chose to use the same cohort of alumni each semester.  By preserving
the cohort, we ensure that the alumni develop a deeper understanding
of the course and what students are and are not capable of.  Alumni
mentors also provide some institutional memory for ongoing projects
as students come and go; that can be particularly helpful if the course
rotates between faculty members.  Of course, there are also benefits in
a rotating group of mentors, including providing a wider range of
experiences and expanding the opportunities for student-alumni
connections.

\subsection{Partnering with career development}

We have found it productive to partner with our career development
office.  The training they give students before meeting with community
partners and the guidance that they give throughout the semester
are invaluable.  For those reasons, we highly recommend 
partnering with your corresponding office.

In the absence of such institutional support, it is still possible to run
your own ``preparing to meet with your community partner'' session.
As we suggest in the section on our work with career development,
it is useful to help students get a broader sense of issues in the
community, to help students unpack their own biases about the
community the likely biases the community has about them, and to
help them figure out the non-technical questions to discuss in the
first meeting with the client.  You will find it useful to brainstorm
with the students to prepare them prepare appropriate questions for
their first partner meetings.

For example, having them consider the different reasons they might
need to contact the client helps them prepare appropriate questions
for that meeting.

\subsection{Identifying community partners}

Getting the right partners is important.  The course requires not
only projects of the right scope and scale, but also partners who
will understand the issues associated with working with a student
development team and who is willing to accept a slow development
process.  You learn many of these things in preliminary meetings
with potential partners; we recommend having such meetings in the
prior semester, if at all possible.  Expect to interview many more
potential partners than the number of projects you need.

We are fortunate to have an outreach coordinator who helps with these kinds
of issues.  Among other things, they are careful to make sure that neither
side overcommits.  Our coordinator is particularly good at saying ``that
seems like too much'', particularly when we discuss the scale of the project
or the time commitments on both sides.  They have also been useful in
helping the partner understand possible projects and accept the limitations
of working with students.

If your institution has someone in this role, we encourage you to
work with them.  If you don't have an outreach coordinator, If not,
it is worthwhile to recruit someone who can serve a similar role
Someone who can provide context, who listens carefully, and who
cautions both sides on over-committing is particularly useful.

\subsection{Time-boxing}

We highly recommend timeboxing, both of individual tasks and of students' 
total hours of work on the project each week.
That said, 
time-boxing is strange to most students.  They are used to assignments 
which demand a particular
product meeting detailed requirements.  You support them in embracing the
time-box model by providing clear expectations about process and quality and
by giving regular feedback, particularly early in the semester.  We have
found weekly individual and group reports to be particularly useful.

\subsection{Contextualizing the course}

As we have suggested, not every student embraces the course.  It
is therefore essential to provide appropriate context for the course at
the beginning of the semester.  Explain to students what you expect them
to get out of the course.  For example, we focus on the benefits of learning
soft skills, of making a difference in the community, of encountering
more realistic situations, of developing the ability to put together a
real product, and of recovering from difficulties.
